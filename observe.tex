\documentclass[a4paper,12pt]{article}

\include{common/data}
\include{common/styles}
\usepackage{setspace} % Можно вручную изменять межстрочный интервал в заданном промежутке текста используя \begin\end рамки


% \title{\Title}

\date{\vspace{-5ex}}  % Удалить дату, автора и пробелы
 \usepackage[left=3cm,right=1.5cm,
    top=2cm,bottom=2cm]{geometry} % Размеры отступов от краёв листа
\usepackage{indentfirst}
\setlength{\parindent}{1.25cm} % Размер отступа абзаца
\linespread{1.25} %Размер межстрочного интервала

 
\begin{document}
\selectlanguage{russian}
\author{}             % Без этой строки LaTeX будет ругаться

\pagestyle{plain}   
\listofnotes          % Список TODO

% \maketitle
\pagenumbering{arabic}  % Нумерация страниц начнется с оглавления

\include{parts/abstract}   % Аннотация
\tableofcontents        % Оглавление

\include{parts/abbreviation}
\include{parts/intro}
\include{parts/standards}
\include{parts/review}
\include{parts/channel}
\include{parts/model}
\include{parts/results}
\include{parts/conclusion}

\printbibliography

\pagebreak
\appendix

% Вставляем дополнение с примерами форматирования в том и только том случае,
% если review = 1 И showExamples = 1. Переменные устанавливаются в файле data.tex.
\ifboolexpr{
    (test {\ifnumequal{\value{review}}{0}}) and
    (test {\ifnumequal{\value{showExamples}}{1}})
}{
    \include{parts/appendix_formatting}
}{}


\end{document}
